\chapter{Ejercicios resueltos}
\label{ApendiceEjercicios}

\section{Ejercicios del Cap\'\i tulo 1}

\begin{exercise}\label{EjercicioCassini} Demostrar la f\'ormula de Cassini:
$$F_{n-1}F_{n+1}-F_n^2=(-1)^n,\ n\geq 2.$$
\end{exercise}

\begin{solution}
Basta tomar determinantes en (\ref{FormulaPotenciaN}). Alternativamente, se puede razonar por inducci\'on:
\begin{eqnarray*}D_n&=&F_{n-1}F_{n+1}-F_n^2=F_{n-1}(F_n+F_{n-1})-F_n(F_{n-1}+F_{n-2})\\ &=&
F_{n-1}^2-F_nF_{n-2}=-D_{n-1}.\end{eqnarray*}
Como $D_2=1$, es claro que $D_n=(-1)^n$.
\end{solution}

\begin{exercise}\label{EjercicioConvergencia} Demostrar que la sucesi\'on $\{F_n/F_{n-1}\}$ es convergente y que su l\'\i mite es $\Phi$.
\end{exercise}

\begin{solution}
Es f\'acil demostrar que si la sucesi\'on es convergente su l\'\i mite $l$ tiene que ser $\Phi$. En efecto, si $n\geq 3$,
$$\frac{F_n}{F_{n-1}}=\frac{F_{n-1}+F_{n-2}}{F_{n-1}}=1+\frac{1}{F_{n-1}/F_{n-2}}.$$
Tomando l\'\i mites obtenemos la ecuaci\'on $l=1+1/l$, es decir, $l^2-l-1=0$ y, necesariamente, $l=\Phi$ (la otra ra\'\i z es $-1/\Phi<0$, luego no puede ser l\'\i mite de la sucesi\'on dada, que es de t\'erminos positivos).

Para demostrar que la sucesi\'on es convergente, demostramos que es de Cauchy. En primer lugar
$$\left|\frac{F_n}{F_{n-1}}-\frac{F_{n+1}}{F_n}\right|=\left|\frac{F_n^2-F_{n-1}F_{n+1}}{F_{n-1}F_n}\right|=\frac{1}{F_{n-1}F_n}\leq \frac{1}{n(n-1)},\ n\geq 6,$$
donde, en la segunda igualdad hemos utilizado la f\'ormula  del Ejercicio \ref{EjercicioCassini} y, para la desigualdad, que $F_n\geq n$ para $n\geq 5$.
Entonces, si $n<m$,
\begin{eqnarray*}\left|\frac{F_n}{F_{n-1}}-\frac{F_m}{F_{m-1}}\right|&=&
\left|\frac{F_n}{F_{n-1}}-\frac{F_{n+1}}{F_n}+\cdots +\frac{F_{m-1}}{F_{m-2}}-\frac{F_m}{F_{m-1}}\right|\\ &\leq& \sum_{i=n}^{m-1}\frac{1}{i(i-1)}=\frac{1}{n-1}-\frac{1}{m-1},\end{eqnarray*}
que es tan peque\~no como se desee si $n$ y $m$ son suficientemente grandes.
\end{solution}

\begin{exercise}\label{ej:Recurrencia}
Sea $\{x_n\}$ una sucesi\'on que satisface la relaci\'on de recurrencia $x_n=4x_{n-1}-4x_{n-2}$ para $n\geq 3$. Expresar $x_n$ en funci\'on de los dos primeros valores $x_1$ y $x_2$.
\end{exercise}

\begin{solution}
Sea $C=\begin{pmatrix}\phantom{-}0&1\\-4&4\end{pmatrix}$. Observamos que, si $n\geq 3$,
$$C\begin{pmatrix}x_{n-2}\\ x_{n-1}\end{pmatrix}=
\begin{pmatrix}x_{n-1}\\ 4x_{n-1}-4x_{n-2}\end{pmatrix}
=\begin{pmatrix}x_{n-1}\\ x_n\end{pmatrix},$$
por tanto
\begin{equation}\label{RelacionMatricial}\begin{pmatrix}x_{n-1}&x_n\\ x_n&x_{n+1}\end{pmatrix}=
C\begin{pmatrix}x_{n-2}&x_{n-1}\\ x_{n-1}&x_n\end{pmatrix}=\dots =
C^{n-2}\begin{pmatrix}x_1&x_2\\ x_2&x_3\end{pmatrix}.\end{equation}
Por otro lado, podemos calcular $C^{n-2}$ expl\'\i citamente si conocemos su forma can\'onica de Jordan. El polinomio caracter\'\i stico de $C$ es $x^2-4x+4=(x-2)^2$ y, si $f$ es el endomorfismo de $\R^2$ con matriz asociada respecto de la base can\'onica $C$, el subespacio propio de $f$ correspondiente al valor propio $2$ es $\langle (1,2)\rangle$. De hecho $(f-2\id)(0,1)=(1,2)$ ($\id$ es el endomorfismo identidad), por tanto se cumple que $P^{-1}CP=J$, donde
$$J=\begin{pmatrix} 2&1\\ 0&2\end{pmatrix}\quad\text{y}\quad P=\begin{pmatrix} 1&0\\ 2&1\end{pmatrix}.$$ Entonces
\begin{eqnarray*}C^{n-2}&=&(PJP^{-1})^{n-2}=PJ^{n-2}P^{-1}=P(2I_2+H)^{n-2}P^{-1}\\&=&
P(2^{n-2}I_2+(n-2)2^{n-3}H)P^{-1}=
\begin{pmatrix}
-(n-3)2^{n-2} & (n-2)2^{n-3}\\ -(n-2)2^{n-1} & (n-1)2^{n-2}\end{pmatrix},\end{eqnarray*}
donde naturalmente $I_2$ denota la matriz identidad $2\times 2$ y $H=\begin{pmatrix}0&1\\ 0&0\end{pmatrix}$. N\'otese que $(2I_2+H)^{n-2}$ puede calcularse mediante la f\'ormula del binomio de Newton, que se simplifica notablemente dado que $H^2=0$.
Ahora, de (\ref{RelacionMatricial}), se obtiene que
$$x_n=-(n-2)2^{n-1}x_1+(n-1)2^{n-2}x_2=2^{n-2}(-2(n-2)x_1+(n-1)x_2).$$
\end{solution}
\begin{remark*} El m\'etodo utilizado en este ejercicio se puede usar con cualquier sucesi\'on definida mediante una f\'ormula de recurrencia lineal. La idea es, dada la f\'ormula de recurrencia $x_n=c_1x_{n-1}+\cdots +c_kx_{n-k}$, $n>k$, considerar la matriz $k\times k$
$$C=\begin{pmatrix}
0&1&0&\dots & 0 &0\\
0&0&1&\dots & 0&0\\
0&0&0&\dots & 0&0\\
\vdots&\vdots&\vdots&\ddots&\vdots\\
0&0&0&\dots&0&1\\
-c_1&-c_2&-c_3&\dots &-c_{k-1}&-c_k
\end{pmatrix},$$
(se llama \emph{matriz compa\~nera} del polinomio $x^k-c_1x^{k-1}-\cdots -c_k$). Razonando como en el ejercicio, resulta que $C^{n-k}$ relaciona una matriz construida a partir de $x_1,\dots ,x_{k+1}$ con otra similar que involucra $x_{n-k+1},\dots ,x_n,\dots,x_{n+k-1}$. Calculando $C^{n-k}$ a trav\'es de su forma can\'onica de Jordan se obtiene el valor expl\'\i cito de $x_n$ en funci\'on de los valores iniciales $x_1,\dots,x_k$. Sobre c\'omo es en general la forma can\'onica de Jordan de la matriz compa\~nera de un polinomio puede consultarse \cite[Teorema 4.3, Corolario 4.4]{Hungerford}.
\end{remark*}

\section{Ejercicios del Cap\'\i tulo 2}

\begin{exercise} \label{ej.tangente} Demostrar que $\tg z=\ctg z-2\ctg 2z$.
\end{exercise}

\begin{solution}
$$\ctg 2z=\frac{1}{\tg 2z}=\frac{1-\tg^2 z}{2\tg z}=
\frac{1}{2}\ctg z-\frac{1}{2}\tg z.$$
El resultado se sigue sin m\'as que despejar $\tg z$.
\end{solution}

\begin{exercise}\label{EjercicioCosecante} Demostrar que $\csec z=\ctg \frac{z}{2}-\ctg z$ y deducir el desarrollo en serie de Laurant de la funci\'on cosecante alrededor de cero.
\end{exercise}

\begin{solution}
$$\ctg z-\ctg 2z=\frac{\cos z}{\sin z}-\frac{\cos^2z-\sin^2z}{2\sin z\cos z}=\frac{1}{2\sin z\cos z}=\csec 2z.$$
Ahora basta cambiar $z$ por $z/2$. Usando esta f\'ormula y (\ref{ctg}) queda
$$\csec z=\frac{1}{z}+\sum_{n=1}^\infty (-1)^{n+1}\frac{b_{2n}}{(2n)!}(2^{2n}-2)z^{2n-1}.$$
\end{solution} 